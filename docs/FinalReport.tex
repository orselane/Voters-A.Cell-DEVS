%! Author = Alex
%! Date = 2025-03-23

% Preamble
\documentclass[11pt]{article}

% Packages
\usepackage{amsmath}

% Document
\begin{document}
% TODO: Title page
\section{Background}
    % TODO title
    \subsection{Abstract Model}
    The Multistate Voter Model is a simple probabilistic cellular automata model.
    %TODO References here?
    Each cell in the model represents a voter, with a defined set of chosen \textbf{preferences}.
    In the voter model, the cells have a 1-radius von Neumann neighborhood.
    The total cell space is 10x10 dimensions.
    The initial preferences of the model is randomly distributed according to a chosen \textbf{distribution}.
    % TODO insert neighborhood diagram here?
    The rules for the next step preference are as follows:
    \begin{enumerate}
        \item For each cycle, a voter's preference can remain unchanged according to some chosen \textbf{probability}, ($u$).
        \item Otherwise, with probability (1-$u$), the voter changes his preference to the preference of one of its neighbors.
        \item The probability of choosing an individual neighbor among its neighborhoods is uniform, which is a probability of (1-$u$)/4.
    \end{enumerate}

    \subsection{Concrete Model}
    For this assignment, the concrete values we have decided for our implementation is as follows:
    \begin{itemize}
        \item \textbf{Preferences:} The set of preferences for a cell (states)
        will be Blue(\textbf{B}), Red(\textbf{B}), Neutral(\textbf{N}).
        \item \textbf{Probability:} The probability constant $u$ will be $0.5$.
        Meaning 50\% of the time, a voter will change their preference to one of their neighbor's.
        \item \textbf{Distribution:} The initial preference distribution will be (B, 45\%), (R, 45\%), (N, 10\%).
    \end{itemize}

    \newpage
    \section{Formal Specifications}
    The formal specification for the atomic Cell-DEVS model is defined as follows:\\
$<X, Y, I, S, $\theta$, N, d, \tau, \delta_{int}, \delta_{ext}, \lambda, ta>$
    % TODO: Make this into a table?
    \begin{itemize}
        \item $S = \{B, R, N\}$
        \item $X = \{x | x \in S\}$
        \item $Y = \{y | y \in S\}$
        \item $N = S^5 \implies \{(1,0)(0,1)(0,0)(-1,0)(0,-1)\} \text{(Von Neumann Neighborhood)}$
        \item d = 1 (time unit)
        \item  I = $< 5, 0, ???, ???>$
%        \item I = $< 5, 0 \{P_1^x, P_2^x,P_3^x,P_4^x,P_5^x\}$
        \item $\theta = (s, phase, \sigma_{queue}, \sigma)$
        \begin{itemize}
            \item $s \in S$
            \item $phase \in \{passive, active\}$
            \item $\sigma_{queue}$ = ???
            % I think sigma represents the time for the next event? I'm not sure why this exists
            \item $\sigma \in R_0^+ \cup \infty$
        \end{itemize}
        % TODO: Format function better
        \item $\tau$ = [$^*$(Value Delay \{ Condition \})]
        \begin{enumerate}
            \item ???
        \end{enumerate}
        \item $\delta_{int} = ???$ % Automated? format it cooler
        \item $\delta_{ext} = ???$ % Automated? format it cooler
        \item $\lambda = ???$ % Automated? format it cooler
        % TODO: Format this function better
        \item ta: \{$\theta_{phase} = active \implies d$ OR $\theta_{phase} = passive \implies \infty$\}
    \end{itemize}



\end{document}